\documentclass[addpoints, answers]{exam}

\usepackage{hyperref} 
\usepackage{CV}
%%%%%%%%%%%%%%%%%%%%%%%%%%%%%%%%%%%%%% % for abbreviations, or acronyms %%%%%%%%%%%%%%%%%%%%%%%
\usepackage[automake, acronym, nopostdot, nohypertypes={acronym,notation}]{glossaries} 
% manual http://tug.ctan.org/macros/latex/contrib/glossaries/glossariesbegin.html
%\usepackage[acronym, nopostdot]{glossaries} 
\makeglossaries %https://tex.stackexchange.com/questions/110095/list-of-acronyms-is-not-displayed

\newacronym{ovh}{OVH}{oral verrucous hyperplasia}
% electric healthcare records (EHR)
\newacronym{ehr}{EHR}{electric healthcare records}

\newacronym{ihc}{IHC}{immunohistochemistry}

\newacronym{hnscc}{HNSCC}{head and neck squamous cell carcinoma}
\newacronym{tcga}{TCGA}{the Cancer Genome Atlas}
\newacronym{tcia}{TCIA}{the Cancer Imaging Archive}
\newacronym{tcpa}{TCPA}{the Cancer Proteome Atlas}

\newacronym{cnn}{CNN}{convolutional neural network}
\newacronym{rnn}{RNN}{recurrent neural network}
\newacronym{gcnn}{GCNN}{graph convolutional neural network} % or graph convolutional network

\newacronym{rna}{RNA}{ribonucleic acid}
\newacronym{rnaseq}{RNA-Seq}{RNA sequencing}
\newacronym{lncrna}{lncRNA}{long non-coding RNA}
%\newacronym{km}{KM}{Kaplan-Meier}
\newacronym{rppa}{RPPAs}{reverse-phase protein arrays}
\newacronym{rpma}{RPMA}{reverse-phase protein lysate microarray}

\newacronym{fdr}{FDR}{false discovery rate}


\newacronym{go}{GO}{Gene Ontology}
\newacronym{ipi}{IPI}{international protein index database}

\newacronym{acta2}{ACTA2}{actin alpha 2, smooth muscle} % or alias as alpha smooth muscle actin,  (α-SMA, SMactin, alpha-SM-actin, ASMA)

\newacronym{mmp}{MMP}{matrix metalloproteinase}
 %DKK1, CAMK2N1, STC2, PGK1, SURF4, USP10, NDFIP1, FOXA2, STIP1, and DKC1
 %ZNF557, ZNF266, IL19, MYO1H, FCGBP, LOC148709, EVPLL, PNMA5, KIAA1683, and NPB

\newacronym{DKK1}{DKK1}{dickkopf WNT signaling pathway inhibitor 1} 
\newacronym{CAMK2N1}{CAMK2N1}{calcium/calmodulin dependent protein kinase II inhibitor 1} 
\newacronym{STC2}{STC2}{stanniocalcin 2} 
\newacronym{PGK1}{PGK1}{phosphoglycerate kinase 1} 
\newacronym{SURF4}{SURF4}{surfeit 4} 
\newacronym{USP10}{USP10}{ubiquitin specific peptidase 10} 
\newacronym{NEDD4}{NEDD4}{neural precursor cell expressed, developmentally down-regulated 4}
\newacronym{NDFIP1}{NDFIP1}{NEDD4 family interacting protein 1} 
\newacronym{FOXA2}{FOXA2}{forkhead box A2} 
\newacronym{STIP1}{STIP1}{stress-induced-phosphoprotein 1} 
\newacronym{DKC1}{DKC1}{dyskeratosis congenita 1, dyskerin} 

\newacronym{ZNF557}{ZNF557}{zinc finger protein 557} 
\newacronym{ZNF266}{ZNF266}{zinc finger protein 266} 
\newacronym{IL19}{IL19}{interleukin 19} 
\newacronym{MYO1H}{MYO1H}{myosin 1H} 
\newacronym{FCGBP}{FCGBP}{Fc fragment of IgG binding protein} 
\newacronym{LOC148709}{LOC148709}{LncRNA LOC148709} 
\newacronym{EVPLL}{EVPLL}{envoplakin-like protein} 
\newacronym{PNMA5}{PNMA5}{paraneoplastic antigen like 5} 
%\newacronym{KIAA1683}{KIAA1683}{IQCN, IQ Motif Containing N} 
\newacronym{IQCN}{IQCN}{IQ motif containing N} % previous name KIAA1683
% "IQ" refers to the first two amino acids of the motif: isoleucine (commonly) and glutamine (invariably)
\newacronym{NPB}{NPB}{neuropeptide B} 
\newacronym{CALML5}{CALML5}{calmodulin like 5}

 \newacronym{rt}{RT}{radiation therapy}
 \newacronym{nccn}{NCCN}{National Comprehensive Cancer Network}
 \newacronym{hif}{HIF}{hypoxia-inducible factor}
 \newacronym{egfr}{EGFR}{epidermal growth factor receptor}
 \newacronym{ras}{RAS}{rat sarcoma}
 \newacronym{hras}{HRAS}{Harvey rat sarcoma viral oncoprotein}
 \newacronym{erk}{ERK}{extracellular signal-regulated kinases}
 \newacronym{us}{US}{United States}
 \newacronym{fda}{FDA}{Food and Drug Administration}
 \newacronym{tpf}{Tax-PF}{docetaxel, cisplatin, and 5-fluorouracil}
 \newacronym{tki}{TKI}{tyrosine kinase inhibitor}
 \newacronym{her}{HER}{human epidermal growth factor receptor}
 \newacronym{ici}{ICI}{immune-checkpoint inhibitor}
 
 
 
 
 %\newacronym{ctla4}{CTLA-4}{cytotoxic T lymphocyte antigen 4}
 \newacronym{pd1}{PD-1}{programmed death 1}
 %\newacronym{pdl1}{PD-L1}{programmed death ligand 1}
 \newacronym{tim3}{TIM-3}{T-cell immunoglobulin mucin protein 3}
 \newacronym{lag3}{LAG-3}{lymphocyte activation gene 3}
 \newacronym{ifng}{IFN-$\gamma$}{interferon gamma}
 \newacronym{tigit}{TIGIT}{T cell immunoglobin and immunoreceptor tyrosine-based inhibitory motif}
 \newacronym{gitr}{GITR}{glucocorticoid-induced tumor necrosis factor receptor}
 \newacronym{vista}{VISTA}{V-domain Ig suppressor of T-cell activation}
 \newacronym{tmsb4x}{TMSB4X}{thymosin beta-4 X-linked}
 \newacronym{emt}{EMT}{epithelial-mesenchymal-transition}
 \newacronym{gdc}{GDC}{Genomic Data Commons}
 \newacronym{nci}{NCI}{the National Cancer Institute}
 \newacronym{ncbi}{NCBI}{the National Center for Biotechnology Information}
 \newacronym{gdac}{GDAC}{Genome Data Analysis Center}
 \newacronym{rest}{REST}{Representational State Transfer} 
 \newacronym{api}{API}{Application Programmable Interface}
 \newacronym{degs}{DEGs}{differentially expressed genes}
 
\newacronym{grch38}{GRCh38}{Genome Reference Consortium Homo sapiens genome assembly 38}
\newacronym{fpkm}{FPKM}{Fragments per kilobase per million reads mapped}
\newacronym{rsem}{RSEM}{RNA-Seq by Expectation-Maximization}
\newacronym{slca}{SLC35E2A}{solute carrier family 35 member E2A}
\newacronym{slcb}{SLC35E2B}{solute carrier family 35 member E2B}
\newacronym{cde}{CDE}{Common Data Element}
\newacronym{id}{ID}{identification}
\newacronym{ajcc}{AJCC}{the American Joint Committee on Cancer}
\newacronym{uicc}{UICC}{he Union for International Cancer Control}
\newacronym{tnm}{TNM}{the tumor size (T), cervical lymph node metastases (N), and distal metastasis status (M)}
\newacronym{ci95}{95\% CI}{95\% confidence interval}
\newacronym{os}{OS}{overall survival}
\newacronym{rfs}{RFS}{recurrence-free survival}
\newacronym{hr}{HR}{hazard ratio}
\newacronym{hpv}{HPV}{human papillomavirus}
\newacronym{ene}{ENE}{extra-nodal extension}
\newacronym{lvsi}{LVSI}{lymph-vascular space invasion}
\newacronym{pni}{PNI}{perineural invasion}
\newacronym{doi}{DOI}{depth of invasion}
\newacronym{lnd}{LND}{lymph node density}
\newacronym{wpoi5}{WPOI-5}{worst pattern of invasion score 5}
\newacronym{glut4}{GLUT4}{glucose transporters 4}
\newacronym{slc2a4}{SLC2A4}{solute carrier family 2 member A4}
\newacronym{trim24}{TRIM24}{tripartite motif-containing 24}
\newacronym{til}{TIL}{tumor-infiltrating lymphocytes}
\newacronym{tmb}{TMB}{tumor mutational burden}

\newacronym{hpa}{HPA}{the Human Protein Atlas}
% The Health Promotion Administration (HPA) in Taiwan
\newacronym{Thpa}{Taiwan HPA}{Taiwan Health Promotion Administration}
\newacronym{cart}{CAR-T}{chimeric antigen receptor T cells}

\newacronym{ptta}{PTTA}{p value of t-test or ANOVA}
\newacronym{anova}{ANOVA}{analysis of variance}
\newacronym{lcms}{LC-MS/MS}{liquid chromatography with tandem mass spectrometry}
\newacronym{maldi}{MALDI MS}{matrix-assisted laser desorption/ionization mass spectrometry}
\newacronym{maldii}{MALDI IMS}{matrix-assisted laser desorption/ionization imaging mass spectrometry}

\newacronym{pcr}{PCR}{polymerase chain reaction}
\newacronym{rtpcr}{RT-PCR}{reverse-transcription PCR}
\newacronym{qpcr}{RT-qPCR}{quantitative real-time reverse-transcription PCR}

\newacronym{tmuh}{TMUH}{Taipei Medical University Hospital}
\newacronym{tmu}{TMU}{Taipei Medical University}

\newacronym{dag}{DAG}{direct acyclic graph}

%ohm$V3V1V2 disserttion
\newacronym{cd}{CD}{cluster of differentiation}
\newacronym{ctla4}{CTLA4}{cytotoxic T-lymphocyte associated protein 4 (CD152)}

\newacronym{icos}{ICOS}{inducible T Cell costimulator}
\newacronym{hvem}{HVEM}{herpes virus entry mediator}
\newacronym{btla}{BTLA}{B and T lymphocyte associated (CD272)}

\newacronym{clsm}{CLSM}{confocal laser scanning microscopy}
\newacronym{dapi}{DAPI}{4’,6-diamidino‐2-phenylindole}
\newacronym{dc}{DC}{dendritic cell}
\newacronym{eb}{EB}{Epstein Barr virus}
\newacronym{ent}{ENT}{ear- nose- and throat,  otorhinolaryngology}
\newacronym{fas}{Fas}{Fas cell surface death receptor (CD95)} % ligand; Fas -> tumor necrosis factor (TNF) family; 
\newacronym{fasl}{FasL}{Fas ligand (CD95L or CD178)} % FasL = CD95L or CD178; 
\newacronym{gphase}{G-phase}{gap phases in mitosis}
\newacronym{gvhd}{GVHD}{graft versus host disease}
\newacronym{hiv}{HIV}{human immunodeficiency virus}
\newacronym{hla}{HLA}{human leukocyte antigen}
%\newacronym{hpv}{HPV}{human papilloma virus}
\newacronym{icd}{ICD}{international classification of diseases}
\newacronym{il}{IL}{interleukin}
\newacronym{lc}{LC}{langerhans cell}
\newacronym{lp}{LP}{lichen planus}
\newacronym{lpl}{LPL}{leukoplakia}
\newacronym{lpldys}{LPL-dys}{leukoplakia with dysplasia but without malignant transformation}
\newacronym{lplca}{LPL‐ca}{leukoplakia with dysplasia with malignant transformation}
\newacronym{mab}{mAb}{monoclonal antibody}
\newacronym{mdsc}{MDSCs}{myeloid-derived suppressor cells}
\newacronym{tam}{TAMs}{tumor-associated macrophages}
\newacronym{caf}{CAFs}{cancer-associated fibroblasts}

\newacronym{mhc}{MHC}{major histocompatibility complex}
%\newacronym{mmp}{MMP}{matrix metalloproteinases}
\newacronym{nkg2d}{NKG2D}{natural killer group 2 member D}
%  MICA MICB is the ligand of NKG2D
\newacronym{mica}{MICA}{MHC-class-I-polypeptide-related sequence A}
\newacronym{micb}{MICB}{MHC-class-I-polypeptide-related sequence B}

\newacronym{nsaid}{NSAID}{non-steroidal anti-inflammatory drugs}
\newacronym{olp}{OLP}{oral lichen planus}
\newacronym{oscc}{OSCC}{oral squamous cell carcinoma}
\newacronym{pdl1}{PD‐L1}{programmed death-ligand 1 (CD274)}
\newacronym{pdl2}{PD‐L2}{programmed death-ligand 2 (CD273)}
\newacronym{pmod}{PMOD}{potentially malignant oral disorder}
\newacronym{ptld}{PTLD}{post‐transplant lymphoproliferative disorder}
\newacronym{pvl}{PVL}{proliferative verrucous leukoplakia}
\newacronym{sir}{SIR}{standard incidence ratio}
\newacronym{sot}{SOT}{solid organ transplantation}
\newacronym{taa}{TAA}{tumor associated antigen}
%\newacronym{tam}{TAM}{tumor associated macrophage}
\newacronym{tcr}{TCR}{T cell receptor}
%\newacronym{til}{TIL}{tumor infiltrating lymphocyte}
\newacronym{tgf}{TGF}{transforming growth factor}
\newacronym{th}{Th}{T helper}
\newacronym{tls}{TLS}{tertiary lymphoid structure}
%\newacronym{tnm}{TNM}{tumor, node, metastasis classification system}
\newacronym{trail}{TRAIL}{tumor necrosis factor-related apoptosis‐inducing ligand}
\newacronym{treg}{Treg}{regulatory T cell}

\newacronym{aldh2}{ALDH2}{aldehyde dehydrogenase 2}

%%%%%%%%%%%%%%%%%%%%%%%%%%%%%%%%%%%%%%%%%%%%%%
\usepackage[comma,authoryear,square]{natbib} % for Havard style citation; sort
%%%%%%%%%%%%%%%%%%%%%%%%%%%%%%%%%%%%%%%%

%This is where you'll edit your class information for each test
\newcommand{\instructor}{\textbf{}}
\newcommand{\class}{\textbf{}} %MOST 2022
\newcommand{\examnum}{\textbf{Subproject Proposal for the Integrated Research Project of Innovative Translational Medicine}}
\newcommand{\examdate}{}
\newcommand{\name}{
\hfill
\textbf{[SP3] PI Name}: \underline{Wu, Chia-Yu}
\underline{Chi, Li-Hsing}}
%\textbf{[SP3] Co-PI Name}: }} % \hspace{3in}

%This is where you can print out your solutions

%\printanswers
\noprintanswers

\begin{document}

% This is where you change the header of the first page and the running header
\pagestyle{headandfoot} 
\firstpageheader{\Large{{\class}}}{\Large{\examnum}}{\instructor}
\runningheader{\class\ - \examnum}{Page \thepage\ of \numpages}{}
\runningheadrule
\runningfooter{Page \thepage\ of \numpages}{}{\hfill{\large{}}}
	% \textbf{Points earned: \makebox[.5in]{\hrulefill} out of \pointsonpage{\thepage} points}
		
\name
%%%%%%%%%%%%%%	
\vspace{3em}
%SP3
{\large Subproject Title: Deep Learning Platform of Depression Analysis with Holistic Healthcare to Predict Malignant Transformation of Oral Verrucous Hyperplasia}\\

\noindent{\rule{\textwidth}{0.4pt}}
    
%This is brief summary of the contents of the exam
	
%This exam contains \numpages\ pages (including this cover page) and \numquestions\ questions.  The total number of  possible points is \numpoints.
According to the National Comprehensive Cancer Network (NCCN) guidelines (Version 3.2021~\citep{Pfister2021}), the current treatment of \acrfull{hnscc} should include of surgery alone, radiation therapy alone, or a combination of the two with adjuvant chemotherapy.
Despite advancements in these therapies, the five-year survival rate for this disease has only marginally improved over the previous decade, and recurrent disease is seen in up to 50\% of all patients~\citep{Forastiere2001,Warnakulasuriya2009}.
During 2008 to 2018 in Taiwan, the one-year survival rate for this disease among males is from 79.56\% to 81.62\%~(statistics from Cancer Registry of \acrlong{Thpa}, available at \url{https://cris.hpa.gov.tw/pagepub/Home.aspx}, accessed on August 2021).
The five-year survival rate for this disease among males ranged from 55.13 percent to 56.03 percent between 2009 and 2014.

Oral verrucous hyperplasia (\acrshort{ovh}) is a premalignant exophytic oral mucosal lesion with a predominantly verrucous or papillary surface; this lesion can subsequently transform into verrucous carcinoma (VC), a well-established warty variant of squamous cell carcinoma (SCC).

Biomarkers of cancer have been utilized for diagnostic, predictive or prognostic purposes. 
A predictive marker helps to select which specific treatment benefits a better survival.
For example, adjuvant radiation therapy should be done after tumor-ablative surgery in positive surgical margin or metastatic lymph node from the neck dissection. %; it  to select a particular treatment over another.
%It guides to determine which further treatment (e.g., ).
In \acrshort{ovh}, a predictive biomarker gives early alarm of its malignant transformation, which prevents unnecessary biopsy repeatedly.
%A prognostic marker infers the natural history of disease (survival), which is independent on treatment of choice. Gene-expression based biomarker can indicate a need for further treatment, but does not help to determine which treatment. 
	
	
\begin{minipage}[t]{5.5in}
\vspace{0pt}
\begin{itemize}
        
% These are the instructions for the class
\item \textbf{Aim 1}: To establish an OVH cohort in \acrfull{tmu} affiliated hospitals, and collect clinical and pathological images of those lesions.  		
\item \textbf{Aim 2}. To construct a deep learning platform of \acrshort{tmu} dataset. A deep-learning-based algorithm might extract features of clinical and pathological whole-slide images of \acrshort{ovh} and \acrshort{hnscc}.
\item \textbf{Aim 3}: Deep Learning-based biomarker identifying in depression analysis with holistic healthcare study. The cortisol is one of the most common objective outcomes studied in psychoneuroimmunological (PNI) factors~\citep{Hulett2016}. Interleukin, cytokines, T-lymphocytes ($CD4^+$, $CD8^+$), $CD56^+$ (neural cell adhesion molecule), and heart rate variability (HRV) are included in other PNI studies.
The subjective psychosocial measurements should include depression, stress, quality of life, anxiety, fatigue, mindfulness, mood, and specific spiritual growth---the meaning of life~\citep{Hsiao2012}. Our hypothesis is that \acrshort{ovh} could be transformed into oral cancer because of the stress, via mind-brain-body network~\citep{Rogers1959}. Thus, besides using those serum biomarkers, we encourage physicians to write their \acrfull{ehr} in a "holistic way".	And the \acrshort{ovh} carriers should use wearable device (e.g., smart watch) to monitor their HRV. Once the \acrshort{ovh} becoming oral cancer, we could early treat the disease as well as analyze the contributing factors of carcinogenesis. That could be a predictive biomarker.
%\item \textbf{Simplify all answers as much as possible.} This means that you need to need to combine like terms, reduce fractions, etc. (You do not need to rationalize denominators.)
	
\end{itemize}

\vspace{3em}
\end{minipage}

\printglossary
%\include{CV}    % cv.tex

\bibliographystyle{agsm.bst} % agsm, IEEEtran
\bibliography{dissertation2022_Tex.bib}

\end{document}


%%%%%%%%%%%%%%%%%%%%%%%%%%%%%%%
\hfill
\begin{minipage}[t]{2in}
\vspace{0pt}
\gradetablestretch{2}
%\vqword{Problem}
\addpoints	
\gradetable[v][questions]
%\gradetable[v][pages]
\end{minipage}

\newpage
\addpoints
